\chapter{Review of Complex Analysis}

We will take some time to review complex analysis in one variable.

\section{Holomorphic functions}

Let \(f\colon U \to \C\) be a function
defined on an open set \(U \subsetopen \C\)
and \(z_0 \in U\) a point.
The \(f\) is \defn{complex differentiable} at \(z_0\)
iff the limit
\[\lim_{z\to z_0} \frac{f(z) - f(z_0)}{z - z_0} \eqqcolon f'(z_0)\]
exists.
Decomposing \(f(z) = f(z_0) + f'(z_0)(z-z_0) + \landauo(z-z_0)\)
and identifying \(\C = \R^2\)
gives us that any complex differentiable function \(f\) is real differentiable
and satisfies \[\diff_{z_0}f(h) = f'(z_0)\cdot h.\]

\begin{prop}
	A function \(f\colon U \to \C\) is complex differentiable
	at \(z_0 \in U\)
	iff \(f\colon U \to \R^2\) is real differentiable at \(z_0\)
	with \(\diff_{z_0}f \in \lin(\R^2, \R^2)\) being \(\C\)-linear.
\end{prop}

\begin{defi}
	A function \(f\colon U \to \C\) is \defn{holomorphic}
	iff \(f\) is complex differentiable at every point in \(U\).
\end{defi}

Thinking of \(U\) as an open subset of \(\R^2\),
we can define \(\partq z \coloneqq \frac12(\partq x - \i\partq y)\)
and \(\partq{\bar z} \coloneqq \frac12(\partq x + \i\partq y)\).
Since \(\partq[f]{\bar z}\) is equivalent to the Cauchy-Riemann equations,
we obtain another characterization of holomorphy.

\begin{prop}
	A function \(f\colon U \to \C\) is holomorphic
	iff \(f\) is real differentiable
	and \(\partq[f]{\bar z} = 0\) vanishes.
\end{prop}

Another characterization arises from the notion of an (oriented) angle.

\begin{defi}
	A function is called \defn{conformal} iff it preserves oriented angles.
\end{defi}

\begin{prop}
	A function is holomorphic iff it is conformal.
\end{prop}

\section{Power series}

% TODO: Power series stuff

\section{Cauchy theory}

A \defn{\(1\)-form} is a function of the form
\(\alpha = a(x, y)\diff x + b(x, y) \diff y\)
for functions \(a, b\colon U \to \C\)
or, equivalently, a function of the form
\(\alpha = u(z)\diff z + v(z)\diff\bar z\)
with \(u, v\colon U \to \C\),
\(\diff z = \diff x + \i\diff y\) and \(\diff\bar z = \diff x - \i\diff y\).
