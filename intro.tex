\chapter{Introduction}

\section{What is a \RS?}

A \RS is a one-dimensional complex manifold.

\begin{defi}
	A \defn{manifold} is a topological space
	which is Hausdorff, second-countable and locally Euclidean
	together with an atlas \(\mathcal A\)
	of charts \(\alpha = (\U\alpha, \uphi\alpha)\)
	consisting of an open set \(\U\alpha\)
	and a homeomorphism
	\[\uphi\alpha\colon \U\alpha \to \R^n\]
	each and
	such that \(M = \bigcup_\alpha \U\alpha\) is covered.

	When the \defn{transition maps}
	\begin{align*}
		\trans\alpha\beta
		\coloneqq \uphi\beta\circ{\uphi\alpha}^{-1}
		\colon
		    \uphi\alpha[\U\alpha\cap\U\beta]
	        \to \uphi\beta [\U\alpha\cap\U\beta]
	\end{align*}
	are all in a class
	\(C \subseteq \{f\colon U \to \R^n \:\vert\: U \subsetopen \R^n\}\),
	we call \((M, \mathcal A)\) a \(C\)-manifold.
\end{defi}

By identifying \(\R^2 = \C\)
and setting
\[	\Hol \coloneqq
	\{f\colon U \to \C \:\vert\:
		\text{\(U \subsetopen \C\),
		\(f\) holomorphic}
	\}
\]
we obtain the definition of a \RS.

\begin{defi}
	A \defn{\RS} is a \(\Hol\)-manifold.
\end{defi}

Since we have
\(\Hol \subset \{f \in \smooth(U, \R^2) \:\vert\: U \subsetopen \R^2\}\),
any \RS is a smooth two-dimensional real manifold,
\ie a surface.

\section{Why is a \RS?}

We give a few, increasingly complex\footnote{pun intended}
examples of \RSs.

\begin{bsp}
Most likely the easiest and most well-known constructions of \RSs
are the following four.
\begin{itemize}
\item \(\C\) is a \RS.
\item Any open subset \(U \subsetopen M\) of a \RS \(M\) is a \RS itself.
\item The Riemann Sphere \(\hat\C = \C \cup \{\infty\} = \S^2\) is a \RS.
\item Suitable quotients of \RSs are \RSs,
	\eg the Torus \(\C / (\Z + \i\Z)\).
\end{itemize}
\end{bsp}

\begin{bsp}
	We can associate \RSs to holomorphic functions.
	For example,
	we would like to define the natural logarithm
	of any \(z = r\e^{\i\theta} \neq 0\)
	awith \(r > 0\) and \(\theta \in \R\) by
	\[
		\ln(z)
		= \ln(r\e^{\i\theta})
		= \ln(r) + \ln(\e^{\i\theta})
		= \ln(r) + \i\theta,
	\]
	However, there is a problem:
	For any \(z \neq 0\),
	the argument \(\theta\) is only well-defined modulo \(2\pi\).
	Choosing any system of representatives
	to obtain a total function \(\ln\colon \C^* \to \C\)
	is neither natural nor will it ever lead to a continuous function.

	This can be fixed in two ways.
	For one, we can think of the logarithm as being multi-valued,
	obtaining
	\[
		\ln(\i)
		= \i\frac\pi2+2\pi\i\Z
		= \left\{
			\frac12\pi\i, \frac52\pi\i,
			-\frac32\pi\i, \frac92\pi\i,
			-\frac72\pi\i, \dots
		\right\}.
	\]
	For the other,
	we need \RSs to “unfold the complex plane”.
\end{bsp}

\begin{bsp}
	Projective non-singular curves are \RSs.
	For example, we can take the polynomial equation
	\[y^2 = x^3+ax+b\]
	with parameters \(a\) and \(b\)
	and study the set of solutions \(S \subseteq \C^2 \subseteq \Proj\C2\),
	where the latter is a holomorphic manifold of complex dimension \(2\).
	In the language of algebraic geometry,
	\(S\) is an algebraic subvariety of \(\Proj\C2\)
	and whenever \(S\) has no singularities,
	it is a \RS.
\end{bsp}

\section{How is a \RS?}

To conclude our initial tour of \RSs,
we consider alternative ways
to look at or define this class of spaces.
A \RS is a real surface with some extra structure.
There are various way to characterize this extra structure:
\begin{description}
\item[As defined] The holomorphic structure can be given with an atlas.
\item[Algebraically] Any compact \RS
	corresponds to a non-singular projective curve.
\item[Conformal] Two Riemannian metrics are considered conformal
	it they only differ by a (pointwise) rescaling
	or~-- equivalently~-- if they measure the same angles.
	A \RS can be equivalently given
	as a real surface together with a conformal class of Riemann metrics.
\item[Hyperbolic] Using the Poincaré uniformization theorem,
	it is possible to show
	that “nearly all” Riemann surfaces are hyperbolic,
	which means that they are quotients of the hyperbolic space \(\H^2\).
\end{description}
