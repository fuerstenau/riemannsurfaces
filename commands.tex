%%% Commands %%%%%%%%%%%%%%%%%%%%%%%%%%%%%%%%%%%%%%%%%%%%%%%%%%%%%%%%%%%%%%%%%%%

% Common sets
\newcommand{\N}{\mathds{N}} % Natural numbers
\newcommand{\Z}{\mathds{Z}} % Integers
\newcommand{\Q}{\mathds{Q}} % Rationals
\newcommand{\R}{\mathds{R}} % Reals
% Overwriting a hyperref command
% We do not know what it would have done
\renewcommand{\C}{\mathds{C}} % Complex numbers

% Common constants
\newcommand{\e}{\mathrm{e}} % Euler's number, exp(1)
\renewcommand{\i}{\mathrm{i}} % Imaginary unit

% Common notations
\newcommand{\abs}[1]{\left\vert#1\right\vert} % Absolute value
\newcommand{\norm}[1]{\left\lVert#1\right\rVert} % Norm
\newcommand{\Po}{\mathcal{P}} % Powerset

% Environments for definitions, theorems etc.
% All of these are numbered together within each chapter
\swapnumbers % Numbers in front, e.g. “1.4 Theorem” instead of “Theorem 1.4”
\theoremstyle{definition}
\newtheorem{defi}{Definition}[chapter]
\newtheorem{bsp}[defi]{Example}
\newtheorem{ex}[defi]{Exercise}

\theoremstyle{plain}
\newtheorem{thm}[defi]{Theorem}
\newtheorem{prop}[defi]{Proposition}
\newtheorem{lem}[defi]{Lemma}
\newtheorem{cor}[defi]{Corollary}

\theoremstyle{remark}
\newtheorem{bem}[defi]{Remark}

% Texty abbreviations
\newcommand*\eg{e.g.\ }
\newcommand*\ie{i.e.\ }

% Semantic commands
\newcommand*\defn[1]{\emph{#1}} % Definiendum

% Useful commands
\newcommand*\mailto[1]{\href{mailto:#1}{#1}}

% Technical helpers
\newcommand*\norml{}
\newcommand*\normr{}
\newcommand*\autol{\!\left}
\newcommand*\autor{\right}
\newcommand*\argl[1]{\csname #1l\endcsname}
\newcommand*\argr[1]{\csname #1r\endcsname}

\makeatletter
\newcommand*\setter[1]{\def\@setter{#1}}
\newcommand*\lecturer[1]{\def\@lecturer{#1}}
\newcommand*\notetaker[1]{\def\@notetaker{#1}}
\makeatother
